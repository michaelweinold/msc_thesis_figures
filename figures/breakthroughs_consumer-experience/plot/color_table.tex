\documentclass[crop]{standalone}

\usepackage{xcolor}
\usepackage{colortbl}

\definecolor{tcs09}{RGB}{233,33,76}
\definecolor{color-daylight}{RGB}{215,67,64}
\definecolor{color-salon_cri90}{RGB}{236,80,69}
\definecolor{color-pfs_cri97}{RGB}{235,69,69}
\definecolor{color-sla_cri82}{RGB}{224,78,67}
\definecolor{color-yag_cri83}{RGB}{221,75,63}
\definecolor{color-qd_cri91}{RGB}{232,77,68}
\definecolor{color-ygag_cri89}{RGB}{229,72,67}
\definecolor{color-258_cri90}{RGB}{231,75,66}

\usepackage{amsmath}
\usepackage{tabularx}

\begin{document}

	\begin{tabularx}{418pt}{|X|X|X|X|X|X|X|X|X|}
		\hline
			\cellcolor{tcs09} \textcolor{white}{R} & 
			\cellcolor{color-daylight} \textcolor{white}{D} &
			\cellcolor{color-pfs_cri97} \textcolor{white}{7} &
			\cellcolor{color-qd_cri91} \textcolor{white}{6} &
			\cellcolor{color-salon_cri90} \textcolor{white}{5} &
			\cellcolor{color-258_cri90} \textcolor{white}{4} &
			\cellcolor{color-ygag_cri89} \textcolor{white}{3} &
			\cellcolor{color-yag_cri83} \textcolor{white}{1} &
			\cellcolor{color-sla_cri82} \textcolor{white}{2} \\
		\hline
	\end{tabularx}

\end{document}