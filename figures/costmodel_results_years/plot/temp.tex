\documentclass[crop]{standalone}

\usepackage{verbatim}

\usepackage{pgfplots}
\usepackage{pgfplotstable}

\usepackage[]{xcolor}

\pgfplotsset{compat=1.8} %v=1.8 required for this plot

% What does this do?
% Number of \prevrow{} is total number of columns minus two
\pgfplotstableset{
    create on use/accumyprev/.style = {
        create col/expr = {	\prevrow{0}+
        					\prevrow{1}+
        					\prevrow{2}+
        					\prevrow{3}+
        					\prevrow{4}+
        					\prevrow{5}+
        					\prevrow{6}+
        					\prevrow{7}+
        					\prevrow{8}+
        					\prevrow{9}+
        					\prevrow{10}+
        					\prevrow{11}+
        					\prevrow{12}+
        					\prevrow{13}+
        					\prevrow{14}+
        					\prevrow{15}+
        					\prevrow{16}+
        					\pgfmathaccuma}
    }
}

\begin{document}

\begin{tikzpicture}

\begin{axis}[
%Grouping
	name=top,
	at={(0,0)},
 %Plot Type
    ybar stacked,
    bar width=15pt,
%Size
    width=595pt,
    height=200pt,
%Domain
    enlarge x limits=0.04,
    ymin=0, ymax=200,
%???
    %point meta = explicit,
%Ticks and Labels
    xtick = data,
%Axes and Labels
    axis on top,
    ylabel = {Cost in Dollarz},
]
% The first plot sets the "baseline":
% Uses the sum of all previous y values,
% except for the last bar, where it becomes 0
% \ifnum\coordindex>n where n is total number of columns minus two
\addplot[
    y filter/.code = {\ifnum\coordindex>16 \def\pgfmathresult{0}\fi},
    draw = none,
    fill = none
] table [col sep=comma, x expr = \coordindex, y = accumyprev] {../data/2003.csv};

\addplot[fill=yellow,draw=black,ybar stacked] table [col sep=comma, x expr = \coordindex, y index = 0, meta index = 0] {../data/2003.csv};
\addplot[fill=lightgray,draw=black,ybar stacked] table [col sep=comma, x expr = \coordindex, y index = 1, meta index = 1] {../data/2003.csv};
\addplot[fill=gray,draw=black,ybar stacked] table [col sep=comma, x expr = \coordindex, y index = 2, meta index = 2] {../data/2003.csv};
\addplot[fill=darkgray,draw=black,ybar stacked] table [col sep=comma, x expr = \coordindex, y index = 3, meta index = 3] {../data/2003.csv};
\addplot[fill=orange,draw=black,ybar stacked] table [col sep=comma, x expr = \coordindex, y index = 4, meta index = 4] {../data/2003.csv};
\addplot[fill=blue,draw=black,ybar stacked] table [col sep=comma, x expr = \coordindex, y index = 5, meta index = 5] {../data/2003.csv};
\addplot[fill=red,draw=black,ybar stacked] table [col sep=comma, x expr = \coordindex, y index = 6, meta index = 6] {../data/2003.csv};
\addplot[fill=red,draw=black,ybar stacked] table [col sep=comma, x expr = \coordindex, y index = 7, meta index = 7] {../data/2003.csv};
\addplot[fill=red,draw=black,ybar stacked] table [col sep=comma, x expr = \coordindex, y index = 8, meta index = 8] {../data/2003.csv};
\addplot[fill=green,draw=black,ybar stacked] table [col sep=comma, x expr = \coordindex, y index = 9, meta index = 9] {../data/2003.csv};
\addplot[fill=green,draw=black,ybar stacked] table [col sep=comma, x expr = \coordindex, y index = 10, meta index = 10] {../data/2003.csv};
\addplot[fill=green,draw=black,ybar stacked] table [col sep=comma, x expr = \coordindex, y index = 11, meta index = 11] {../data/2003.csv};
\addplot[fill=white,draw=black,ybar stacked] table [col sep=comma, x expr = \coordindex, y index = 12, meta index = 12] {../data/2003.csv};
\addplot[fill=yellow,draw=black,ybar stacked] table [col sep=comma, x expr = \coordindex, y index = 13, meta index = 13] {../data/2003.csv};
\addplot[fill=yellow,draw=black,ybar stacked] table [col sep=comma, x expr = \coordindex, y index = 14, meta index = 14] {../data/2003.csv};
\addplot[fill=yellow,draw=black,ybar stacked] table [col sep=comma, x expr = \coordindex, y index = 15, meta index = 15] {../data/2003.csv};
\addplot[fill=yellow,draw=black,ybar stacked] table [col sep=comma, x expr = \coordindex, y index = 16, meta index = 16] {../data/2003.csv};
\addplot[fill=yellow,draw=black,ybar stacked] table [col sep=comma, x expr = \coordindex, y index = 17, meta index = 17] {../data/2003.csv};
% Plots the connecting lines
% Number of \thisrow{} is number of connecting lines
% Number of \thisrow{} is total number of columns minus one
  \addplot [
    const plot, black,
    point meta = {
        TeX code symbolic = {
            \pgfkeys{/pgf/fpu/output format=fixed}
            \pgfmathtruncatemacro\upperbound{
                \thisrowno{0}+
                \thisrowno{1}+
                \thisrowno{2}+
                \thisrowno{3}+
                \thisrowno{4}+
                \thisrowno{5}+
                \thisrowno{6}+
                \thisrowno{7}+
                \thisrowno{8}+
                \thisrowno{9}+
                \thisrowno{10}+
                \thisrowno{11}+
                \thisrowno{12}+
                \thisrowno{13}+
                \thisrowno{14}+
                \thisrowno{15}+
                \thisrowno{16}+
                \thisrowno{17}
            }
            \edef\dostuff{
                \noexpand\def\noexpand\pgfplotspointmeta{%
                    \thisrowno{0}--\upperbound%
                }
            }%
            \dostuff
        }
    },
  ] table [col sep=comma, x expr = \coordindex, y expr = 0] {../data/2003.csv};

\end{axis}

\begin{axis}[
%Grouping
    name=bottom,
	at={(top.west)},
	yshift=-200pt,
	anchor=west,
%Size
	height=150pt,
	width=538pt,
%Domain
	xmin=1,xmax=20,
	ymin=0,ymax=1,
%Type
	%ybar stacked,
	bar width=15pt,
	axis on top,
%Ticks
    enlarge x limits=0.0,
    scaled y ticks = false,
	xtick=data,
	xticklabels={,,},
	yticklabels={0,25,50,75,100},
	ytick={0,0.25,0.5,0.75,1},
	/pgf/number format/1000 sep={},
%Axis Labels
	ylabel style={align=center},
%	ylabel = Yield \text{[}\%\text{]},
%General
    axis on top,
%Legend
	legend cell align={left},
	legend style={at={(axis cs:0.8,0.13)},anchor=south west},
			]
\addlegendimage{line legend, red}
\addlegendentry{Total Yield}
\addlegendimage{/pgfplots/refstyle=bar1,xshift=0.5em}
\addlegendentry{Step Yield}

\addplot [ybar,ybar legend,draw=blue,fill=blue!30]
	table [col sep=comma, x=index, y=yield]
	{../data/2003_yield.csv}; \label{bar1}
\addplot [no markers, red]
	table [col sep=comma, x=index, y=cumyield]
	{../data/2003_yield.csv};
\end{axis}

%Labels only
\begin{axis}[
%Grouping
	%Grouping
    name=bottom1,
	at={(top.west)},
	yshift=-200pt,
	anchor=west,
%Size
	height=150pt,
	width=538pt,
%Domain
	xmin=1,xmax=20,
	ymin=0,ymax=1,
%Type
	%ybar stacked,
	bar width=15pt,
	axis on top,
%Ticks
    enlarge x limits=0.0,
    scaled y ticks = false,
	xtick=data,
	xticklabels={,,},
	yticklabels={,,},
	ytick={0,0.25,0.5,0.75,1},
	/pgf/number format/1000 sep={},
%Other Labels
    visualization depends on={value \thisrow{repeats} \as \labela},
    nodes near coords={$\times$\labela},
%General
    axis on top,
%Bottom Labels
    xticklabel style={rotate=90, align=right, anchor=east},
    xticklabels = { Epitaxy,
                    Wafer Inspection,
                    Dry Deposition,
                    Litho (Stepper),
                    Litho (Mask),
                    Dry Etch,
                    ITO Depn.,
                    N-Metal,
                    P-Metal,
                    Contact Metal,
                    Generic Metal,
                    Wafer Bonding,
                    Wafer LLO,
                    Backgrinding,
                    CMP,
                    Dicing,
                    Probe Test,
                    TOTAL},
    ]
\addplot [no markers, only marks]
	table [col sep=comma, x=index, y=labelheight]
	{../data/2003_yield.csv};
\end{axis}
    
\end{tikzpicture}

\end{document}
