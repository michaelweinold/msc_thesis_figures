\documentclass[crop,tikz]{standalone}

\usepackage{../../../supplementary/header/header_pgfplot_standalone}

\begin{document}
\begin{tikzpicture}
%%%%%%%%%%%%%%%%%%%%%%%%%%%%%%%%%%%%%%%%%%%%%%%%%%

\begin{axis}[
%Grouping
	name=rebel,
	height=5cm,
	width=4cm,
%Domain
	ymin=0, ymax=1,
	xmin=400, xmax=750,
%Axis Labels
	xlabel style = {align=center},
	ylabel=Intensity \text{[}Arb.Units\text{]},
	xlabel=Wavelength \text{[}nm\text{]} \\ Philips Lumiramic,
			]
\addplot [red]
	table [col sep=comma, x=wavelength, y=sensitivity]
	{../data/luminosity.csv};
\addplot []
	table [col sep=comma, x=wavelength, y=intensity]
	{../data/lumileds_lumiramic.csv};
\end{axis}

\begin{axis}[
%Grouping
	name=narrow,
	at={(rebel.west)},
	xshift=3cm,
	anchor=west,
	height=5cm,
	width=4cm,
%Domain
	ymin=0, ymax=1,
	xmin=400, xmax=750,
%Ticks
	yticklabels={,,},
%Axis Labels
	xlabel style = {align=center},
	xlabel=Wavelength \text{[}nm\text{]} \\ Philips Narrow,
			]
\addplot [red]
	table [col sep=comma, x=wavelength, y=sensitivity]
	{../data/luminosity.csv};
\addplot []
	table [col sep=comma, x=wavelength, y=intensity]
	{../data/luxeon_narrow.csv};
\end{axis}

\begin{axis}[
%Grouping
	name=trigain,
	at={(narrow.west)},
	xshift=3cm,
	anchor=west,
	height=5cm,
	width=4cm,
%Domain
	ymin=0, ymax=1,
	xmin=400, xmax=750,
%Ticks
	yticklabels={,,},
%Axis Labels
	xlabel style = {align=center},
	xlabel=Wavelength \text{[}nm\text{]} \\ GE TriGain,
			]
\addplot [red]
	table [col sep=comma, x=wavelength, y=sensitivity]
	{../data/luminosity.csv};
\addplot []
	table [col sep=comma, x=wavelength, y=intensity]
	{../data/ge_trigain.csv};
\end{axis}

\begin{axis}[
%Grouping
	name=qd,
	at={(trigain.west)},
	xshift=3cm,
	anchor=west,
	height=5cm,
	width=4cm,
%Domain
	ymin=0, ymax=1,
	xmin=400, xmax=750,
%Ticks
	yticklabels={,,},
%Axis Labels
	xlabel style = {align=center},
	xlabel=Wavelength \text{[}nm\text{]} \\ Osram QD,
			]
\addplot [red]
	table [col sep=comma, x=wavelength, y=sensitivity]
	{../data/luminosity.csv};
\addplot []
	table [col sep=comma, x=wavelength, y=intensity]
	{../data/osram_qd.csv};
\end{axis}


%%%%%%%%%%%%%%%%%%%%%%%%%%%%%%%%%%%%%%%%%%%%%%%%%%
\end{tikzpicture}
\end{document}
